\documentclass{article}
\usepackage{amsmath, amsthm, amssymb}
\usepackage{listings}
\usepackage{graphicx}
\usepackage{float}
\usepackage{enumerate}
\usepackage{fancyhdr}
\usepackage[labelfont=bf]{caption}
\usepackage[left=0.75in, top=1in, right=0.75in, bottom=1in]{geometry}
\pagestyle{plain}
\begin{document}
\rhead{Aaron Okano, Anatoly Torchinsky, Samuel Huang, Justin Maple \\ 
      ECS 132: Homework 1}
\thispagestyle{fancy}

% The list environment is just to get some vertical spacing
\list{} \item \endlist

% Let the homework begin!
\section*{Problem 1} 

\subsection*{(a)}

We know that there are $2^{3} = 8$ total number of outcomes because each of the
three coins can take on only one of two values. Only three of those outcomes
$(W = 20, W = 30, W = 40)$ are at least 20. The probability of a single
one of those outcomes is $\frac{1}{8}$, so $P(W \geq 20)=P( W = 20 \cup W=30
\cup W=40 )=P( W = 20 ) + P( W = 30 ) + P( W = 40 ) = 3 \cdot \frac{1}{8} =
.375$.

\subsection*{(b)}

From (a) we know that there are only three outcomes where $W \geq 20$, and only
two of them include nickels---$W = 20$ and $W = 40$. Therefore, the probability
that your friend got a nickel given he won at least 20 cents is $\frac{2}{3}$.

\section*{Problem 2}

\subsection*{(a)}

Let us consider the case where the passenger does not succeed in making the
bus. Also let A be the delay of bus A and B be the delay of bus B. Then, $P(A -
B \geq 1.25)$ is the probability that the passenger misses bus B. This can be
split into two cases: where $B = 0$ or $B = 1$. Then, $P( A - B \geq 1.25 ) =
P( B = 0 \cap [A = 2 \cup A = 3] ) + P( B = 1 \cap [A = 3]) = P(B=0)\cdot [P(A
= 2 ) + P( A = 3 ) ] + P(B=1)\cdot P(A = 3 ) = 0.5\cdot (0.15 + 0.1) + 0.25\cdot
0.1 = 0.15$. Now the probability that the passenger succeeds is $1 - 0.15 =
0.85$.

\subsection*{(b)}

Among the successful transfers, we must now find the probability that A arrives
before B, which can be represented as $P( B - A > 0 \vert A - B \leq 1.25 )$.
According to (2.6) from the book and from part (a), we can write this as
$\frac{ P( (B - A > 0) \cap (A - B \leq 1.25) ) }{ P( B - A > 0 ) }$. Since the
first event in the numerator only exists if the second is true, this equation
can be simplified to $\frac{ P( B - A > 0 ) }{ P( A - B \leq 1.25 ) }$. From
part (a) we know what $P( A - B \leq 1.25 )$ is. The numerator is $P(B = 1)
\cdot P(A = 0) + [P(B = 2)\cdot (P(A = 0) + P(A = 1))] + [P(B = 3) \cdot (P(A=0) +
P(A=1) + P(A=2))]$. The numerator evaluates to 0.3275 and the denominator is
0.85. Therefore, the probability is $\frac{0.3275}{0.85} = .385$.

\end{document}
